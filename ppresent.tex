\documentclass{beamer}

\usepackage{mytheme}

% For Title Page
% ---------------------------------------------------------------------------------------

\title{Applications of Galois Theory}

\author[Sandesh Thakuri]{Mr. Sandesh Thakuri}

\institute[CDM, TU]
{Central Department of Mathematics, TU}

\date{2080-6-4}


% Begin Document
% ---------------------------------------------------------------------------------------------------------------------


\begin{document}
\myfootline

\begin{frame}[plain]
  \tikzonlytitlepage
  \titlepage
\end{frame}

\begin{frame}{Contents}
  \tableofcontents
\end{frame}

\small
\section{Introduction}
\begin{frame}{Background}

  \textcolor{blue}{\textbf{1.Field Extension}}\\[2mm]
  A field E is said to be an extension field of a field F denoted by E/F, if F can be embedded in E.\\[8mm]

  \textcolor{blue}{\textbf{2.Galois Field Extension}}\\[2mm]
  The field E of F is said to be the Galois extension if E is normal extension of F.\\[8mm]

  \textcolor{blue}{\textbf{3.Galois Group}}\\[2mm]
  Let E/F. Then the set of all \underline{automorphisms} of E that
  fixes F, denoted by Aut(E/F) forms a group under the function
  composition. This group is called the Galois Group.

\end{frame}
\end{document}



%%% Local Variables:
%%% mode: latex
%%% TeX-master: t
%%% End:
