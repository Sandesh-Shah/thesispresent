\begin{frame}{Galois Groups of Cubic}
\begin{definition}[Discriminant of a Polynomial]
  Let \(f \in K[x]\) a polynomial of degree \(n\) with \(n\) distinct roots \(u_1,u_2,...,u_n\) in some splitting field \(F\) of \(f\) over \(K\). Let \\
  \(\Delta = \prod\limits_{i<j}(u_i-u_j) = (u_1-u_2)(u_1-u_3)...(u_{n-1}-u_n) \in F\).\\
  The discriminant of \(f\) is the element \(D= {\Delta}^2\) \cite{hunger}.
\end{definition}
\vspace{5mm}

\begin{theorem}[Theorem]
  If \(f\) is a separable polynomial of degree \(3\), then the Galois group of \(f\) is \(A_3\) if and only if the discriminant of \(f\) is the square of an element of \(K\) \cite{hunger}.
\end{theorem}
\end{frame}

\begin{frame}{Galois Group of Quartics}
  \begin{definition}[Resolvant Cubic of a Quartic]
Let \(u_1,u_2,...,u_4\) be the roots of a quartic \(f \in K[x]\) and\\ \(\alpha=u_1u_2+u_3u_4,\) \(\beta=u_1u_3+u_2u_4,\) \(\gamma=u_1u_4+u_2u_3\). \\[3mm]
\textcolor{green!50!black}{The cubic polynomial whose roots are \(\alpha,\beta,\gamma \)} is called the resolvant cubic of \(f\) which is a polynomial over \(K\) \cite{hunger}.
\end{definition}
\vspace{2mm}
\textbf{\textcolor{blue}{An Application of the Fundamental Theorem}} \\[3mm]
Let \(V=\{(1),(12)(34),(13)(24),(14)(23)\} \in S_4\).\\[2mm]
Now under \textcolor{green!50!black}{the Galois correspondence the subfield \(K(\alpha, \beta, \gamma)\) corresponds to the normal subgroup \(G \cap V\)} \cite{hunger} because \(K(\alpha,\beta,\gamma)\) is a splitting field of the resolvant cubic
whose Galois group is a subgroup of \(S_3\) and only normal subgroup \(N\) of \(S_4\) with \(|N| \leq 6\) is \(V\),\\[2mm]
\textcolor{violet}{Hence \(K(\alpha, \beta, \gamma)\) is Galois over \(K\) and \(Aut_K^{K(\alpha, \beta, \gamma)} = G/(G \cap V)\) \cite{hunger}}.
\end{frame}

\begin{frame}
  \begin{tcolorbox}[colback=white, colframe=blue!40, boxsep=0mm]
\begin{theorem}[Theorem]
  Let \(K\) be a field and \(f \in K[x]\) a separable quartic with Galois Group \(G\). Let \(\alpha, \beta, \gamma\) be the roots of the resolvant cubic of \(f\) and let\\
  \textcolor{magenta}{\(m= [K(\alpha, \beta, \gamma) : K]\)} then,
\begin{enumerate}
\item[i)] \(m=6 \Longleftrightarrow G=S_4\);
\item[ii)] \(m=3 \Longleftrightarrow G=A_4\);
\item[iii)] \(m=1 \Longleftrightarrow G=V\);
\item[iv)] \(m=2 \Longleftrightarrow G=D_4\) or \(G={\mathbb{Z}}_4\); in this case \(G={\mathbb{Z}}_4\) if \(f\) is irreducible over \(K(\alpha, \beta, \gamma)\) and \(G={\mathbb{Z}}_4\) otherwise\cite{hunger}.
\end{enumerate}
\end{theorem}
\end{tcolorbox}
\vspace{2mm}

\begin{theorem}[Theorem]
If \(p\) is a prime and \(f\) is an irreducible polynomial of \textcolor{violet}{degree \(p\)} over \(\mathbb{Q}\) which has precisely \textcolor{violet}{two non-real roots}, then the Galois group of \(f\) is \textcolor{violet}{\(S_p\)} \cite{hunger}.
\end{theorem}
\end{frame}

\begin{frame}[allowframebreaks]{Illustrations}
  Galois theory gives the precise condition under which a polynomial of degree \(n \geq 5\) is solvable by radicals or not.

  \vspace{7mm}
   \begin{tcolorbox}[colback=white, colframe=brown!80!black, boxsep=3mm, title={\bfseries \color{white} Example}]
  The polynomial is \textcolor{green!50!black}{\(x^5-1 \in \mathbb{Q}[x]\)}.\\
  The set of  roots of this polynomial are the fifth roots of unity which forms a group under addition modulo \(5\). Hence the \textcolor{violet}{Galois group is isomorphic to \(\mathbb{Z}_5\)} \cite{hunger}. The group \(\mathbb{Z}_5\) is cyclic and ``every cyclic group is solvable'' \cite{galois}. Hence this polynomial is solvable by radicals.
\end{tcolorbox}


%%% Local Variables:
%%% mode: latex
%%% TeX-master: "<none>"
%%% End:
